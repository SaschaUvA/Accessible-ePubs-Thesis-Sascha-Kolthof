%%%%%%%%%%%%%%%%%%%%%%%%%%%%%%%%%%%%%%%%%%%%%%%%%%%%%%%%%%%%%%%%%%%%%%%%%%%%%%%%
% setup.tex
% Main tex file for setup and content collocation
% For the use of University of Amsterdam
% Information Systems and Data Science students
% Adapted by Riccardo Fiorista (riccardo.fiorista@proton.me)
%%%%%%%%%%%%%%%%%%%%%%%%%%%%%%%%%%%%%%%%%%%%%%%%%%%%%%%%%%%%%%%%%%%%%%%%%%%%%%%%

% Options:
% Choose one of:
%% `is` - Information Systems
%% `ds` - Data Science 
% Add (separated by `,`):
%% `nolinenumbering` - If you want to remove line numbering on submission
%% `draftmargins` - If you would like to give your reviewer more space for comments
%% `nofrontpicture` - If you do not wish to have a graphic on your front-page
%% `nofirstcompanypicture` - If you do not wish to have a graphic on your front-page
%% `nosecondcompanypicture` - If you do not wish to have a graphic on your front-page
\documentclass[ds, nofrontpicture, nolinenumbering, nosecondcompanypicture, nofirstcompanypicture]{mscthesis}

%%%%%%%%%%%%%%%%%%%%%%%%%%%%%%%%%%%%%%%%%%%%%%%%%%%%%%%%%%%%%%%%%%%%%%%%%%%%%%%%
% DOCUMENT METADATA
%%%%%%%%%%%%%%%%%%%%%%%%%%%%%%%%%%%%%%%%%%%%%%%%%%%%%%%%%%%%%%%%%%%%%%%%%%%%%%%%

% Thesis related entries
\title{AccessiPub: Automating EPUB Accessibility}
\subtitle{Introducing a new method to increase EPUB accessibility automatically}

% Date on which your thesis is submitted
\date{30.06.2024}

% 4-5 keywords should do the trick. They should ideally be phrases of 2-4 words or single words.
\keywords{EPUB, accessibility automation, WCAG, HTML refactoring, tool development}

% Author data
\authorname{Sascha Kolthof}
\authorid{12851981}
\authoremail{sascha.kolthof@student.uva.nl}

% Supervisors
\uvasupervisorname{Yixian Shen}
\uvasupervisoraffiliation{UvA Supervisor}
\uvasupervisoremail{y.shen@uva.nl}

% % Comment if you do not have an external supervisor
% \externalsupervisorname{External Supervisor} \externalsupervisoraffiliation{External Supervisor}
% \externalsupervisoremail{supervisor@company.nl}

% % Uncomment and fill paths if you want to add custom images
% %% Figure size suggestions (in general it's best to render them from SVGs):
% %% 3000x3000 @ 240dpi for all three
% \titlepicturepath{}
% \firstcompanypicturepath{}
% \secondcompanypicture{path}

%%%%%%%%%%%%%%%%%%%%%%%%%%%%%%%%%%%%%%%%%%%%%%%%%%%%%%%%%%%%%%%%%%%%%%%%%%%%%%%%
% CONTENT
%%%%%%%%%%%%%%%%%%%%%%%%%%%%%%%%%%%%%%%%%%%%%%%%%%%%%%%%%%%%%%%%%%%%%%%%%%%%%%%%

\begin{document}
\pagestyle{plain}

\maketitlepage
\fixemptypage
\setcounter{page}{1}

\begin{abstract}
\thispagestyle{plain}
The EPUB file format is an open-to-use file format with many accessibility perks. For visually impaired people, EPUBs can provide great benefits. EPUBs work well with assistive technologies and can adapt dynamically to alterations such as font size and style changes. 
Unfortunately, many EPUBs do not adhere to the most recent guidelines put in place by the EPUB Accessibility standard. Current academia has not yet explored the possibility of automating the process of making EPUBs more accessible. 
This thesis therefore formulates a new methodology, called AccessiPub, to enhance EPUB accessibility automatically. The Ace checker is leveraged, along with new methods to locate accessibility errors inside the EPUB. Then, drawing from existing literature, HTML refactoring is adapted to fit the EPUB file type. Lastly, a novel way to disassemble and reassemble EPUBs is created using EbookLib and ZipFile. 
The developed methodology was able to reduce the amount of target errors by 97.8\%. Compared to similar research, this result is very promising and lights up the path ahead for further research into automatic EPUB accessibility enhancement.
\end{abstract}

\maketitle

% UNCOMMENT THIS LINE BEFORE EXPORTING FINAL (?)
\section*{GitHub Repository}
\small
\url{https://github.com/SaschaUvA/Accessible-ePubs-Thesis-Sascha-Kolthof}
\normalsize

% Sections; Try to stick to this setup but you can comment each section
\section{Introduction}
\label{sec:introduction}
Ebooks are a great way of making regular books more portable and convenient. An EPUB (shorthand for \textit{electronic publication}) is an open-to-use file format for ebooks, developed by the International Digital Publishing Forum (IDPF), that focuses on reflowable content. Reflowable content is a concept where alterations, like increasing the font size, make content on the page flow dynamically to accommodate this change. For people with special needs, this offers new ways for books and other literature to be more accessible. EPUBs are also less cumbersome to read and transport for people with visual impairments than bulky braille books or books with large fonts. 

Additionally, certain accessibility tools can leverage the EPUB file structure, making it the preferred option for people with disabilities \cite{Kasdorf2018}. This can include: text-to-speech devices (screen readers), math-specific screen readers, and braille displays \cite{Giusti2016, Kasdorf2018, Kim2019}.
EPUBs have enormously higher accessibility potential compared to other file formats, such as PDFs. Bartalesi and Leporini explored this in their study, by evaluating 25 users, of which 18 were visually impaired. They concluded that 88\% of the tested users preferred the EPUB format over the PDF format \cite{Bartalesi2015}. Since then the EPUB specification has evolved even further in providing good accessibility.

The COVID-19 pandemic forced people to stay at home and tend to online libraries instead of physical libraries. This increased the amount of EPUB resources in libraries, simultaneously highlighting the issue with the current state of EPUB accessibility \cite{Chee2022}. Specifically, many EPUBs currently do not adhere to the latest accessibility standards posed by the EPUB Accessibility standard. This standard has been created under the name of the World Wide Web Consortium (W3C), the same entity that brought forth the Web Content Accessibility Guidelines (WCAG). 

While some research exists on what has to be done to improve the current accessibility landscape for EPUBs, most fail to go into how to make these improvements \cite{Chee2022}. Others, like Campoverde-Molina et al. \cite{Campoverde2020} propose a manual approach by urging educational institutions to undertake projects that improve the accessibility of their resources. This is not a viable solution for the problem, due to the large amount of EPUBs being available. More crucially for this thesis, current academia has not explored automating the process of making EPUBs more accessible so that it can be done on a large, cost-effective scale.

This thesis aims to fill this gap and do a first venture into this specific area by formulating a new methodology to automatically make EPUBs more accessible. This complete start-to-finish pipeline is named AccessiPub. Multiple steps need to be taken in order to formulate this new methodology. First, a way of locating accessibility errors within EPUBs is drawn up with the help of the Ace checker. Then, we draw from existing literature on increasing web content accessibility due to the relation between EPUBs and regular web content. This investigation into the existing literature yields HTML refactoring, a method that we adapted to fit EPUBs. These alterations are integrated with a novel way of using EbookLib and ZipFile to disassemble and reassemble an EPUB's content files. These methods are combined to form a start-to-finish pipeline. This pipeline can take an EPUB, analyze it for fixable errors, and repair these, while making sure to retain the integrity of the EPUB. This leads to the main research question of this thesis: \\

\noindent\textit{To what extent can EPUB accessibility errors automatically be repaired to improve the accessibility of an EPUB without harming its original integrity?} \\

\noindent This research question will be answered with the help of the following five subquestions:
\begin{itemize}
    \item How can accessibility errors be located inside an EPUB exactly?
    \item How can HTML refactoring be adapted to fit the EPUB file format?
    \item How can the modified content be reconstructed into a functioning EPUB?
    \item To what extent does the devised methodology reduce accessibility errors in EPUBs?
    \item To what extent can the original integrity of the EPUB be upheld?\\
\end{itemize}

\noindent The first three subquestions are methodological, while the latter two will be answered by the experiments. 
\section{Related Work}
\label{sec:related_work}
This thesis aims to address the existing research gap concerning the potential for automatically refactoring EPUBs to enhance their accessibility. This section will explore the current research surrounding this topic by focusing on three topics. The first subsection will explore the most recent issue of the WCAG due to its strong ties to the EPUB Accessibility standard. The subsequent subsection will then discuss the additions that the EPUB Accessibility standard brings on top of the WCAG. The last subsection will focus on current methods for automatically increasing web content accessibility.

Since this is a new field of research, related work on this topic is sparse. Few research articles exist with a similar goal of increasing accessibility automatically. The foundational concepts and theories on the subject of EPUB accessibility will be discussed and related to the subject of web content accessibility. If we know how EPUB accessibility relates to web content accessibility, we can apply the techniques used for automating web content accessibility on EPUBs. These techniques will underpin the chosen methods in this thesis and later, will help answer the three methodological subquestions.


\subsection{Web Content Accessibility Guidelines}
Before we can go in-depth on the specifications of the EPUB Accessibility standards, the WCAG is discussed first. The most recent version of the Web Content Accessibility Guidelines at the time of writing is 2.2. It is important to note that each version since 2.0 does not replace the old version, but rather adds additional guidelines on top of the former version. For version 2.2, this means that nine new success criteria were added compared to 2.1 \cite{WCAGGuidelines}. The WCAG consists of four main principles: Perceivable, Operable, Understandable, and Robust \cite{Caldwell2008}. Since new versions only apply incremental changes, the base structure and principles behind WCAG 2 have remained the same \cite{WCAGGuidelines}.

\textit{Perceivable} is the first principle listed by the WCAG and lays its focus on pursuing perceivable UI components and other web content. This includes guidelines, such as providing alternative text for images, making content distinguishable through color or contrast, and captions for time-based media \cite{Caldwell2008}.

Secondly, the WCAG states that navigating the web content and other UI components must be \textit{operable} to all. This principle is comprised of being able to control the web content by keyboard alone, providing enough time for users to read all text, and not having flashing elements that can cause a seizure \cite{Caldwell2008}.

The third principle of the WCAG is \textit{understandable}. This principle states that UI components and information on the page should be understandable for each user. This means, for example, that the web page should operate predictably to the user's input and that input fields are labeled when input is required \cite{Caldwell2008}.

Lastly, it is stated that content must be \textit{robust} to changes in order to be able to be offered to all users. This includes the usage of assistive technologies that rely on the inner structure of the HTML files that make up the EPUB. This means that each element has opening and closing tags (e.g. <p> and </p>), and that all IDs are unique \cite{Caldwell2008}.

The WCAG also specifies conformance levels. These levels are A, AA, and AAA, increasing in level of accessibility demands. While level A is considered to be the minimum required level, it remains deficient in meeting the expectations of users with disabilities significantly. This is why level AA is generally used as the standard of accessibility \cite{KBDaisyWCAG}. The AAA conformance level does not fall within the scope of this thesis. This level includes success criteria such as prerecorded sign language, making this widely regarded to not be essential in providing good accessibility, while still being good practice for accessible publications \cite{WCAGGuidelines, KBDaisyWCAG}.



\subsection{The EPUB Accessibility standard}
Flowing forth from the WCAG, the EPUB Accessibility standard arises. The first 1.0 specification appeared on the 5th of January 2017 under the name of the IDPF; just weeks before this foundation was absorbed into the W3C. The current EPUB Accessibility 1.1 specification, which came out on the 25th of May 2023, was then also released under the W3C name. This ties it strongly to the WCAG, which is developed by the same entity. The standard was developed in strong collaboration with the creators of the Ace checker, the DAISY consortium.

In relation to the WCAG, the EPUB Accessibility standard holds many of the same core values. One of the main requirements in the EPUB Accessibility standard is that EPUBs must conform to at least the WCAG 2.0 A conformance level \cite{EPUBGuidelines}. Since the WCAG already describes web content accessibility, the additions of the EPUB Accessibility standard focus on what further requirements are needed when it comes to collections of web documents \cite{EPUBGuidelines}. The first requirement is that text-audio playback should be synchronized. This specific requirement falls outside of this thesis' scope, as we do not focus on the auditory experience \cite{EPUBGuidelines}. 

The second requirement is proper page navigation. These should be in place so that a visually impaired person can still easily jump from page to page. Proper page navigation consists of page breaks indicating a new page that are structured in a page list. The page list forms the main way of navigating the EPUB for a visually impaired person. Lastly, the pagination source should be defined, which helps the user identify from where the page breaks originate; for example a specific edition of the book \cite{EPUBGuidelines}. These two requirements combined, form the main addition to the WCAG.



% \subsection{Manual methods of EPUB accessibility enhancement}
% - Look at maybe manual methods? What do they do? and can we take from it?

% This subsection could be before or after the next one



\subsection{Automatic methods for web accessibility}
This subsection will explore the existing literature on automatically improving accessibility on web pages. Since EPUBs are in essence a collection of (X)HTML files, this is the closest relevant and related research in this area. This thesis builds on this existing literature, as it is the current state-of-the-art for any automatic methods of improving web content accessibility. Furthermore, these articles will be used to compare our results to at the end of this thesis.

Ferati \& Sulejmani propose possible techniques for automatically improving accessibility on web pages\cite{Ferati2016}. Their research poses an automated tool that focuses on employing three techniques. Firstly, link enrichment is used to detect if a hyperlink is missing an appropriate label and adds an \textit{aria-labelledby} attribute to the link in order to make it visible for screen readers \cite{Ferati2016}. Furthermore, the image enrichment step provides alt-text to images that do not have that already. On images that contained text, Optical Character Recognition was applied to generate alt-text. In case Optical Character Recognition was not applicable to an image, the Google Image API was used to fetch the description of similar images \cite{Ferati2016}. Lastly, Ferati \& Sulejmani applied navigation enrichment, which skips the navigation bar at the top of a web page. This way, screen readers will not continually repeat the navigation bar each time the page is reloaded \cite{Ferati2016}. All techniques use a form of HTML refactoring to apply the accessibility changes to the web page. This combination of techniques was able to resolve 48\% of all target errors \cite{Ferati2016}.

Another approach is proposed by Ikhsan \& Candra, employing an automatic refactoring method \cite{Ikhsan2018}. Their approach consisted of a small number of steps, keeping the overall process simple. First, certain HTML tags were found using the proprietary HTML Code Sniffer. Then, this HTML tag would be checked for a certain condition. If this condition is met, the tag is compliant, otherwise, a pre-defined solution is applied to this tag. An example of such a condition is: if a \texttt{<head>} tag is found without a title attribute, this attribute is created and added to the  \texttt{<head>} tag\cite{Ikhsan2018}. This reduced the number of errors by between 53.6\% and 98.2\% depending on the website \cite{Ikhsan2018}. However, the research does not mention how, for instance, the content of the title attribute is created. It is thus possible that the overall accessibility of the web page did not improve significantly, in defiance of the significant reduction in errors.

Lastly, Zhang et al. make use of the Beautiful Soup 4 (BS4) library in their approach. The objective of their research is to make SVG buttons on web pages more accessible by providing alt-text \cite{Zhang2024}. One of the challenges that Zhang et al. faced, was that the SVG buttons are hard to identify on a web page. This specific challenge was tackled using BS4, employing its ability to find all potential SVG elements, and then sifting through these to find the legitimate SVG button. This strategy made it possible to provide alt-text to SVG buttons. The refactoring approach of Zhang et al. reduced the amount of their target error by 91.9\% \cite{Zhang2024}.
\section{Methodology}
\label{sec:methodology}
This section will showcase the methodology of this thesis and how AccessiPub was developed. The first subsection aims to justify the chosen methods and will highlight what we add to the existing methodology. After this, the used data will be examined along with a presentation of the chosen target errors for this thesis. This is followed by an extensive exposition of the experimental setup where the chosen tools are discussed in further detail. Finally, the evaluation of the results will be discussed.

Currently, there is no existing research on the automation of enhancing EPUB accessibility. In order to tackle the four main challenges that the methodology addresses, existing methods from different fields were called upon and integrated into a new approach, putting forth a framework for a new pipeline to be used for enhancing EPUB accessibility automatically. These four main challenges consist of the following:\\

\begin{enumerate}
\item Disassembling the EPUBs into HTML files
\item Finding the errors in the HTML content of the EPUB
\item Refactoring the HTML files
\item Reassembling the EPUB into a working ebook, without harming the integrity of the original work.\\
\end{enumerate}


\subsubsection{EPUB disassembly}
\noindent To disassemble the EPUBs into their constituent HTML files, the EbookLib library in Python was used. This library is able to read EPUB files and decompose them into its constituent parts, such as images, chapters, and a table of contents. It also contains a powerful feature-set which proves to be important during the reassembly of the EPUB. EbookLib has been used in other scientific literature as well \cite{SefaraTextExtraction, VirinchiBrailleCloud}. Sefara et al. and Virinchi et al. have used EbookLib mainly to extract information from EPUBs. This thesis adds to this by using this tool for a new goal and to reassemble the EPUB at a later stage.


\subsubsection{Error finding}
The next challenge was to find the errors in the HTML files of the EPUB. The errors within the EPUBs were tracked down using the aforementioned Ace checker. This open-source tool, developed by the Daisy Consortium, checks an EPUB file for breaches of built-in accessibility rulesets. The Ace checker uses rulesets that are based on the EPUB Accessibility standard. This is further enhanced by their strong collaboration in the creation of this standard. The exact specifications can be found on the Ace checker project's GitHub page \cite{AceRules}. 

There is no prior research that employs Ace to find the exact location of identified errors within an EPUB. This thesis therefore poses another addition to the field by putting forth a technique that finds the location of the errors identified by Ace. Regular expressions, in combination with BS4, are used to accomplish this. Regular expressions are useful for this purpose, because they allow for great tuning of their specificity. This precision is what allows regular expressions to target only the necessary snippets of the HTML code, altering the source file as little as possible in order to safeguard the integrity of the original author’s work.


\subsubsection{HTML refactoring}
Prior research has used HTML refactoring in order to improve the accessibility of web pages\cite{Ikhsan2018,Ferati2016,Zhang2024}. HTML refactoring is a technique of directly changing the source code of an HTML file, in this case, to improve accessibility. This technique is used for this thesis as well. Ikhsan \& Candra and Ferati \& Sulejmani use their own proprietary system for finding accessibility errors in web pages \cite{Ikhsan2018, Ferati2016}. Their research makes no explicit mention of specific code libraries for HTML refactoring, but both follow the same structure for their strategy. First, the errors are found and assessed whether they are apt for refactoring, then finally, the refactoring is applied \cite{Ikhsan2018, Ferati2016}. The research of Zhang et al. makes explicit mention of employing the BS4 library for HTML refactoring \cite{Zhang2024}. It follows the same basic steps as Ikhsan \& Candra and Ferati \& Sulejmani, but uses BS4 to find all of the elements that cause the errors and refactor them.

Sefara et al. and Virinchi et al. also use the BS4 library for a similar goal. While their purpose is not to increase the accessibility of web pages, it still shows how BS4 can be used for HTML refactoring \cite{SefaraTextExtraction, VirinchiBrailleCloud}. These articles use BS4 mainly to extract information from HTML files stemming from EPUBs specifically. The current academia supports that BS4 is a good tool to use for HTML refactoring and it is used in this thesis for this purpose as well. It is able to find elements inside an HTML page by searching for a specific tag. For example, the entire collection of \texttt{<a>} tags, representing links, can be found and iterated over. Upon finding a tag, its attributes can be read and altered, after which the altered HTML soup can be easily saved back to an HTML file. This thesis adds to the current methods for HTML refactoring by using BS4 to enhance the accessibility of EPUBs, a new target format.


\subsubsection{EPUB reassembly}
After the HTML refactoring has taken place, the altered HTML files are repacked into a working EPUB again. This step has not been documented in scientific literature, so this is a new addition on its own. However, since EbookLib has been used to disassemble the EPUBs, it also serves as the chosen tool to reassemble them. During the reassembly phase, EbookLib allows for the automatic creation of a page list which helps in the effort of improving EPUB accessibility. Reassembly with the EbookLib library does cause certain issues related to missing \texttt{<head>} tags in HTML files and lost metadata in the .opf file. A custom patching pipeline was designed with the help of the ZipFile library to remedy these issues. The experimental setup subsection will go into more detail about how EbookLib is used for this exactly and how the use of the ZipFile library aims to patch these errors.
    

\subsubsection{Complete pipeline framework}
Combining all previously described steps into a pipeline presents the primary novel addition to the field. This pipeline is able to be applied to many other EPUB accessibility problems, providing a methodological base for further endeavors.



\subsection{Data}
\subsubsection{Data sources}
The data for this thesis consists of the OAPEN Foundation collection (Open Access Publishing in European Networks). The foundation aims to provide open-access peer-reviewed titles \cite{OAPENHome}. This collection was provided at the start of the project by the OAPEN Foundation. This dataset was analyzed by the Ace checker to gain insights into the amount of errors present in the dataset.

A secondary dataset, sourced from the Standard Ebooks collection was considered earlier on in the project. The Standard Ebooks LLC is a volunteer-based initiative to provide ebooks in the realm of the public domain \cite{StandardEbooksAbout}. After analyzing the dataset, it became clear that it contained nearly no accessibility errors and none of the errors within this thesis' scope. It was therefore dropped from being included in the tool development and evaluation. However, it provided insights into how accessible EPUB files are constructed and is thus still valuable to be highlighted in this section.


\subsubsection{Data description}
The OAPEN Foundation collection consists of 938 ebooks. The average number of pages is 270 when rounded to a whole number. The OAPEN EPUBs were passed through the Ace checker in bulk, providing insights on the number of accessibility errors in the ebooks. The following error classifications with their respective amounts were found and laid out in Table \ref{table:errorclassesamounts}.

The DAISY consortium describes four different classifications of errors as can be seen in Table \ref{table:errorclassesamounts}. The following definitions are taken straight from the DAISY consortium \cite{AceImpact}:
\begin{itemize}
    \item Minor errors are considered to be a nuisance or an annoyance and should only be fixed if the fix only takes a few minutes\cite{AceImpact}.
    \item Moderate errors result in some difficulty for people with disabilities, but will generally not prevent them from accessing fundamental features or content\cite{AceImpact}.
    \item Serious errors result in serious barriers for people with disabilities, and will partially or fully prevent them from accessing fundamental features or content\cite{AceImpact}.
    \item Critical errors result in blocked content for people with disabilities, and will prevent them from accessing fundamental features or content\cite{AceImpact}.
\end{itemize}

\begin{table}[h!]
% \small
\begin{center}
\begin{tabular}{ | c | c |} 
\hline
\textbf{Error classification} & \textbf{Count} \\
\hline
\textit{Minor} & 94,993 \\
\hline
\textit{Moderate} & 2,674 \\
\hline
\textit{Serious} & 682,133 \\
\hline
\textit{Critical} & 99 \\
\hline
\hline
\textit{\textbf{Total}} & 779,899 \\
\hline
\end{tabular}
\end{center}
\normalsize
\caption{Error prevalence per classification}
\label{table:errorclassesamounts}
\end{table}


\subsubsection{Target errors}
This thesis does not focus on all errors found by the Ace checker, but rather narrows down to three target errors to keep the scope focused. This thesis focuses on the three most prevalent errors that are classified as serious by the Ace checker. These three errors are the following: \texttt{epub-pagelist-missing-
pagebreak}, \texttt{epub-pagelist-broken} and \texttt{link-in-text-block}. 

The first two errors are tied to the page navigation requirement from the EPUB Accessibility standard. The \texttt{epub-pagelist-missing-
pagebreak} error declares that the page list must reference all page breaks. Without this, visually impaired people will not be able to use to page list to navigate to all page breaks. The \texttt{epub-pagelist-
broken} error arises when the page list references something that is not a page break. This can prevent visually impaired people from fully relying on the page list for navigation, since it also points to other elements than page breaks.. Lastly, the \texttt{link-in-text-block} error occurs when links are not distinguishable by means other than color. Especially people with color blindness are impacted by this.

While serious errors are not as impeding as critical errors, the three target errors impact a large number of ebooks, while critical errors do not impact many ebooks. Just 21 ebooks are impacted by a critical error, while 509 out of the 938 ebooks, or 54.3\%, are impacted by at least one of the target errors. As is shown by Table \ref{table:datatargeterrors}, the target errors make up 97.2\% of all serious errors and 85.0\% of the total amount of errors in the entire dataset.

\begin{table}[h!]
% \small
\begin{center}
\begin{tabular}{ | c || c | c | c | c | c |} 
\hline
\textbf{Error} & \textbf{Count} \\
\hline
\textit{link-in-text-block} & 620,203 \\ 
\hline
\textit{epub-pagelist-broken} & 31,372 \\
\hline
\textit{epub-pagelist-missing-pagebreak} & 11,436 \\
\hline
\hline
\textbf{\textit{Total}} & 663,011 \\
\hline
\end{tabular}
\end{center}
\normalsize
\caption{Target errors; top 3 errors classified as serious by the Ace checker}
\label{table:datatargeterrors}
\end{table}


\subsubsection{EPUB file format}
An EPUB file is in essence a zipped folder containing four main parts. The main content of the EPUB is found in the collection of (X)HTML files. These files represent the different sections of the ebook, like the cover, introduction, and chapters. Another part of the EPUB is the folder that holds the images. The CSS style sheet is also stored separately. Finally, there is a collection of files that hold the structure of the EPUB. These structure files consist of a table of contents, metadata, and an index file that holds the file types of all included files. The structural files make sure that the EPUB file is read and interpreted correctly by an e-reader.

\subsection{Experimental setup}
This subsection will cover the experimental setup of the developed methodology. The process is visualized in Figure \ref{figure:flowchart} and can be split up into six steps:\\

\begin{enumerate}
  \item The EPUB is checked with Ace to detect errors.
  \item The EPUB is opened and disassembled with EbookLib.
  \item Ruleset breaches are found with BS4 and RegEx.
  \item Ruleset breaches are mended according to their error type.
  \item The EPUB is reassembled with EbookLib.
  \item Errors induced by this flow are patched
\end{enumerate}

\begin{figure}[h!]
\includegraphics[width=0.48\textwidth,keepaspectratio]{media/images/Flowchart_Methodology_V2.png}
\caption{Process overview of the developed methodology}
\centering
\label{figure:flowchart}
\end{figure}


\subsubsection{Error detection}
To detect the errors that this method addresses, the EPUB is first checked with the Ace checker. The Ace checker generates a JSON and HTML report that represents the assertions of any failed tests. The JSON report is parsed and analyzed to check if any of the three target errors are detected. Depending on the outcome, two booleans (PageBreakFix and LinkFix) are set that dictate which of the respective EPUB fixes are applied. The value of the PageBreakFix boolean is determined by the presence of the \texttt{epub-pagelist-missing-pagebreak} and \texttt{epub-pagelist-broken} errors and the LinkFix value is dependent on whether the \texttt{link-in-text-block} error is found. If neither boolean is set to True, the program closes.


\subsubsection{EPUB preparation}
In the case that an EPUB violates one of the aforementioned rules, the EPUB is loaded into a Python script. This is done with the \texttt{read\_epub()} function from the EbookLib library. The full EPUB data is now held inside an \texttt{EpubBook} object, subsisting of chapters represented as \texttt{EpubHtml} objects, images as \texttt{EpubImage} objects, and other file types as \texttt{EpubItem} objects.


\subsubsection{Error tracing}
Before the \texttt{link-in-text-block} error can be fixed, the specific HTML snippets that cause the error need to be found in the chapters of the EPUB. This is done to alter the source file as little as possible in order to safeguard the integrity of the original author's work. To find the HTML snippets, the JSON report from Ace is loaded which holds the data of the errors. This error data also holds the HTML snippet of the error, denoting its location within the EPUB. For each chapter, this violating snippet of the HTML code is captured in a dictionary datatype. This dictionary now holds all violating snippets per chapter.

After this, the violating snippets are refined further to allow them to match to their respective links in the next step. This is necessary because the HTML snippets generated by the Ace report contain noise making them unable to match to the original HTML code. This rather noisy HTML code from the Ace output can be seen in Figure \ref{figure:aceexample}. The refinement is done by finding the full value of the \texttt{href} attribute using regular expressions. The exact regular expression pattern used for this is: \texttt{href=".*?"}. This results in an organized dictionary with refined HTML snippets that can be used in the next step.


\begin{figure}[h!]
\includegraphics[width=0.48\textwidth,keepaspectratio]{media/images/aceexample.png}
\caption{Example of an Ace \texttt{link-in-text-block} error with an unrefined HTML snippet}
\centering
\label{figure:aceexample}
\end{figure}


\subsubsection{Error fixing}
After the violating HTML snippets have been refined and organized per chapter, the \texttt{link-in-text-block} errors in the EPUB can be fixed. This is done by looping over the HTML files in the EbookLib \texttt{EpubBook} object. Using the \texttt{get\_items\_of\_type
(ebooklib.ITEM\_DOCUMENT)} command, only the HTML files and thus chapters, within the EPUB are targeted. If no violating links are present in a chapter, the chapter can be skipped. However, if violating links are indeed present in the chapter, the HTML snippets are loaded by calling the dictionary. The chapter's content is then parsed with the BS4 library into a 'soup' by using the built-in BS4 'html-parser'.

\begin{figure*}[h!]
\includegraphics[width=0.96\textwidth,keepaspectratio]{media/images/refactorexample2.png}
\caption{Example of how HTML refactoring directly changes HTML code and improves the accessibility of an EPUB}
\centering
\label{figure:refactorexample}
\end{figure*}

BS4 is used to find all links - represented as \texttt{<a>} tags - in the parsed chapter content by using the \texttt{find\_all()} function, similar to the method of Zhang et al.\cite{Zhang2024}. These links are then looped over and checked if their \texttt{href} attribute is matched with one from the dictionary. If this is the case, it is assigned a \texttt{style} attribute as follows: \texttt{link['style'] = "text-decoration:underline"}. Assigning the value at this level makes sure that the style is applied and visible. Inline HTML styling always takes precedence over styling in the stylesheet, given that the value in the CSS stylesheet does not have the \texttt{!important} tag. In case the \texttt{'text-decoration'} for links is set to 'none' in the stylesheet of the EPUB and contains an \texttt{!important} tag, the tag is removed. This is done with the cssutils library by looping through all style rules and filtering on the aforementioned conditions. Once a style rule is found that matches all conditions, the \texttt{!important} tag is removed.

Following the complete link patch of the HTML content of a chapter, it is converted into a bytestring format using the built-in \texttt{bytes()} function. This is necessary for the EPUB as it also works with bytestrings. This bytestring object is then set as the new content of the \texttt{EpubHtml} object, directly modifying its content, highlighting another strong point of the EbookLib library. All chapters are iterated and patched in this way until the process is complete.


\subsubsection{EPUB reassembling}
After the previous step is completed, the EPUB has to be reassembled. This process is not straightforward, as the EbookLib library is not built to make edits to existing books, but rather to create new books from scratch \cite{EbookLibFAQ}. This limitation causes some hindrances and requires workarounds to make sure that the EPUB is reassembled properly. 
First, a new \texttt{EpubBook} object is created. The title, author, and other metadata are copied over from the original EPUB. All items from the original EPUB are then iterated through. All items that have not been modified are copied over directly to maintain the ebook's integrity. If an item has been modified in the previous step, it is ensured that the modified version is copied over. After this, the original table of contents and spine are copied. The \texttt{EpubNcx()} and \texttt{EpubNav()} functions from EbookLib are then used, as these create the new page-list by using the new modified content. In the case of an \texttt{epub-pagelist-missing-pagebreak} or \texttt{epub-pagelist-broken} error, these functions resolve these errors. The new and updated \texttt{EpubBook} object is now saved to a new EPUB file.

It is important to note in this step that a crucial change was made on line 100 in the utils.py file in the source code of EbookLib. This part of the code is responsible for generating the new page list as part of the \texttt{EpubNcx()} and \texttt{EpubNav()} functions. Previously, this line stated "\texttt{if 'epub:type' in elem.
attrib:}" and has been changed to "\texttt{if elem.get('epub:type') ==
'pagebreak':}". The earlier code only checks if an 'epub:type' is present as an element's attribute. However, this code should be more specific since multiple values for this exist. If the epub:type attribute is set to 'footnote' for example, it should not be included in the page list. The new code makes sure that only page breaks are included in the page list.


\subsubsection{EPUB patching}
This flow unfortunately raises several new errors, mainly caused by limitations in EbookLib mentioned previously. This final step focuses on patching these errors. The tool that is used to accomplish this is the ZipFile module, from the ZipFile library. This library is also able to disassemble an EPUB file, but lacks the powerful functionality that the EbookLib library possesses. This simplicity, however, is what enables the EPUB file to have its elements altered without the errors, caused by EbookLib, to appear again. The newly updated EPUB is loaded into a ZipFile object, allowing it to be looped over in Python.

When reassembling an EPUB with EbookLib, an issue with the library causes it to not include the information stored within the \texttt{<head>} tag \cite{EbookLibHeadTag}. 
This leads to the stylesheet link being lost along with the title, severely altering the appearance of the EPUB. The conceived workaround is to store the information that is inside the \texttt{<head>} tag of the original EPUB. This is added to the new file by looping over all files that end in .xhtml, .html, or .htm. Every file's content is converted to a BS4 object, allowing the \texttt{<head>} tag to be replaced. After, the altered content is converted back into bytes and saved back to the file.

% A similar issue as the above causes the schema

Another issue that plagued EPUBs with a missing page-list is their lack of a reference to a source of the page-list, which is another Ace rule breach and ties in with the EPUB Accessibility standard. The pagination source is stored inside the EPUBs .opf file; the file that hosts most of the metadata of the EPUB. The ZipFile object is looped over to find the .opf file and the content is parsed with BS4. This allows for two \texttt{<meta>} tags and one \texttt{<dc:source>} tag to be added, holding the pagination source information. The pagination source is declared as this thesis' tool name: \textit{AccessiPub}.

Lastly, a large fraction of EPUBs are missing an explicit mention of their language in the .opf file, despite the fact that it's declared in their metadata. If the missing value is detected, the \texttt{xml:lang} attribute is set by taking the language from the EPUB's metadata. This is done during the same step as adding the page source since the BS4 object of the .opf file is already loaded. The BS4 object is converted back to bytes and saved to the .opf file, completing the EPUB patching.

With these additional patches in place, the EPUB is saved to a new file by calling ZipFile's \texttt{close()} function, completing the experimental setup.

\begin{table*}[t!]
% \scriptsize
\begin{center}
\begin{tabular}{ | c || c | c | c |}  
\hline
\textbf{Error type} & \textbf{\#Errors Pre-AccessiPub} & \textbf{\#Errors Post-AccessiPub} & \textbf{\% Change} \\
\hline
link-in-text-block & 620,203 & 14,028 & -97.7\% \\
\hline
epub-pagelist-broken & 31,372 & 0 & -100\% \\
\hline
epub-pagelist-missing-pagebreak & 11,436 & 695 & -93.9\% \\
\hline
\hline
Total target errors & 663,014 & 14,723 & -97.8\% \\
\hline
Total errors & 779,899 & 128,194 & -83.6\% \\
\hline
\end{tabular}
\end{center}
\normalsize
\caption{Target error reduction before and after AccessiPub intervention}
\label{table:targetreduction}
\end{table*}

\subsection{Evaluation}
To evaluate the performance of the methodology, a baseline comparison is conducted. The Ace checker is run before and after the experiment to generate accessibility reports of both states. The following key metrics were taken into account to measure the performance of AccessiPub: total EPUB error reduction, total target error reduction, and relative error reduction per EPUB.
Both the total EPUB error reduction and the total target error reduction were calculated using the following formula, converting the resulting fraction to a percentage. The relative error reduction per EPUB was calculated with the same formula, but applied on a per-EPUB basis.

\[E_{reduction} = \frac{E_{pre AccessiPub} - E_{post AccessiPub}}{E_{pre AccessiPub}}\]


To further solidify the results, a paired t-test was performed to acquire insight into the statistical significance (p < 0.05). The evaluation is conducted on the full OAPEN Foundation dataset and aims to show to what extent the devised methodology reduces accessibility errors in EPUBs.

For this thesis' results, there is no direct comparison possible to other scientific papers, since no other methods exist for improving EPUB accessibility automatically. In order to still put the results of this thesis within the frame of other research, they are compared to the results of papers that improve the accessibility of web pages. In particular, the performance of HTML refactoring on EPUBs is compared to the performance of HTML refactoring on web pages.
\section{Results}
\label{sec:results}
The experiment described in the experimental setup has been performed on the 938 EPUBs from the OAPEN Foundation dataset. This section will present the results that have come forth from this. In line with the latter two subquestions of this thesis, the subsections will cover the target error reduction and the original EPUB integrity, followed by a brief exposition of the computational overhead.

\subsection{Target error reduction}
The three target errors show a significant reduction after AccessiPub intervention. The results are visualized in Table \ref{table:targetreduction}. The largest reduction is for the \texttt{epub-pagelist-broken} error, which sees a complete 100\% reduction in errors after AccessiPub intervention. The \texttt{link-in-text-block} error is reduced by 97.7\%, and lastly, the \texttt{epub-pagelist-missing-pagebreak} error is reduced by 93.9\%. The total amount of target errors saw a reduction of 97.8\%, meaning an 83.6\% reduction in the total amount of errors in the dataset.


\subsubsection{Error reduction distribution}
To assess how AccessiPub performs over the whole dataset, the distribution of the reduction of errors is plotted in a histogram, visible in Figure \ref{figure:reduction}. The first bin of the histogram shows that many EPUBs do not have their errors reduced. A further analysis reveals that 438 EPUBs have zero error reduction. 429 of these EPUBs do not have any of the target errors, explaining the lack of error reduction, as these have not been processed by AccessiPub. This still leaves 9 EPUBs with target errors, that did not see any error reduction.
The histogram's far-right edge shows that many EPUBs still have their errors reduced significantly. Out of the 509 EPUBs with at least one of the target errors, 498 had all target errors eliminated. This large reduction in target errors meant a more than 90\% reduction in total errors for 344 EPUBs.

\begin{figure}[h]
\includegraphics[width=0.48\textwidth,keepaspectratio]{media/images/fancy_reduction.png}
\caption{Distribution histogram of the number of EPUBs per error reduction percentage after AccessiPub intervention}
\centering
\label{figure:reduction}
\end{figure}

\subsubsection{Unprocessed EPUBs}
As the analysis of Figure \ref{figure:reduction} showed, 9 EPUBs with a target error did not see any reduction. A deeper dive into these 9 EPUBs shows the cause behind this. AccessiPub failed to process these EPUBs for three different reasons. 

Three EPUBs caused an error due to an incompliant file structure. The cover images were placed in the root folder of the EPUB, where only a mimetype file, a metadata folder, and a content folder should be present according to the EPUB3 standard. This caused a parsing error in the Ace checker, thus not being able to complete the evaluation, despite properly being processed by AccessiPub.

Two other EPUBs had a video represented in the EPUB manifest as a link to a cloud-hosted file. When EbookLib tried to disassemble this EPUB, it expected this video to be physically present in the content folder. However since it was hosted in the cloud, it was not found and EbookLib raised an error and was unable to continue. This is a known issue for the EbookLib developers\footnote{https://github.com/aerkalov/ebooklib/issues/281}.

The last four EPUBs that failed, did so because of a missing table of contents file. This missing table of contents caused EbookLib to generate an empty NCX file, since the generation of it depends on a table of contents. This empty NCX file caused a parsing error when EPUB reassembly was attempted. This is also a known issue for EbookLib\footnote{https://github.com/aerkalov/ebooklib/issues/130}.


\subsubsection{Incomplete target error reduction}
Out of the 509 EPUBs with at least one of the target errors, 498 EPUBs had all target errors eliminated. This means that 11 EPUBs did not have their target errors fully removed. 9 of these EPUBs were not processed as per the paragraph above, leaving 2 EPUBs with incomplete target error reduction. Further investigation reveals that both EPUBs were left with a singular \texttt{link-in-text-block} error. This appears to be falsely reported by Ace, as the link is visible as an accessible link when rendering in a browser. Evidence of this false error for both EPUBs can be found in Appendix \ref{sec:apx:first_appendix}. A GitHub issue for the Ace developers has been created\footnote{https://github.com/daisy/ace/issues/421}. This analysis allows us to report that, apart from unprocessed EPUBs and a bug in the Ace checker, AccessiPub was successful in removing all target errors.


\subsubsection{Statistical significance}
A paired samples t-test was performed to compare the number of errors in the full dataset before AccessiPub intervention and after AccessiPub intervention.
There was a significant difference in the amount of errors between the full dataset before AccessiPub intervention (M = 831.45, SD = 1112.11) and after AccessiPub intervention (M = 136.67, SD = 322.34); t(937) = 21.060, p = 6.461e-81.

% Additionally, a paired samples t-test was also performed on the target error dataset before AccessiPub intervention and after AccessiPub intervention.
% This evaluation also showed a significant difference in the amount of errors between the target error dataset before AccessiPub intervention (M = 1458.60, SD = 1173.41) and after AccessiPub intervention (M = 178.57, SD = 381.54); t(508) = 27.164, p = 5.000e-101.

% \begin{figure}[h]
% \includegraphics[width=0.4\textwidth,keepaspectratio]{media/images/TR_reduction.png}
% \caption{Distribution histogram of the reduction of errors per book after AccessiPub intervention on the data with the target errors}
% \centering
% \label{figure:tr_reduction}
% \end{figure}

\subsection{Original EPUB integrity}
The original EPUB integrity can be analyzed with the help of the visualization in Table \ref{table:integrity}. It shows that the overall amount of errors across all error classifications does not increase. A slight decrease is even seen for minor and moderate errors, which has to do with how EbookLib generates a new table of contents. The main decrease is still seen for the serious errors, since the target errors are all serious errors. 


\begin{table}[h!]
\small
\begin{center}
\begin{tabular}{ | c || c | c | c |} 
\hline
\textbf{Error classification} & \textbf{Before} & \textbf{After} & \textbf{\% Reduction} \\
\hline
Minor & 94,993 & 94,512 & -0.5\% \\
\hline
Moderate & 2,674 & 2,587 & -3.3\% \\
\hline
Serious & 682,133 & 30,996 & -95.5\% \\
\hline
Critical & 99 & 99 & 0\% \\
\hline
\hline
Total errors & 779,899 & 128,194 & -83.6\% \\
\hline
\end{tabular}
\end{center}
\normalsize
\caption{Overall error reduction per Ace error classification before and after AccessiPub intervention}
\label{table:integrity}
\end{table}

While the overall amount of errors did indeed not increase, certain individual errors did increase. This is visualized in Table \ref{table:errorincrease}. The first error describes that each HTML document must contain a non-empty \texttt{<title>} element. The second error describes a faulty order in the table of contents. The latter five errors all relate to missing metadata in the .opf file that describes an EPUB's accessibility characteristics. When taken all together, this increase in errors is very small. In total 18 errors are added to the grand total of 128,360 after AccessiPub intervention. 

\begin{table}[h!]
\small
\begin{center}
\begin{tabular}{ | c || c | c | c |} 
\hline
\textbf{Error type} & \textbf{Before} & \textbf{After} & \textbf{Increase} \\
\hline
document-title & 463 & 466 & 3 \\
\hline
epub-toc-order & 3 & 12 & 9 \\
\hline
metadata-accessibilityfeature & 637 & 639 & 2 \\
\hline
metadata-accessibilityhazard & 654 & 655 & 1 \\
\hline
metadata-accessibilitysummary & 645 & 646 & 1 \\
\hline
metadata-accessmode & 638 & 639 & 1 \\
\hline
metadata-accessmodesufficient & 638 & 639 & 1 \\
\hline
\end{tabular}
\end{center}
\normalsize
\caption{Error increase for certain error types before and after AccessiPub intervention}
\label{table:errorincrease}
\end{table}

The errors do indicate that the EPUB patching step has not gone without errors. The first error indicates that information inside the \texttt{<head>} tag has gone missing for three HTML documents across three separate EPUBs. The latter five errors point to the fact that the missing accessibility metadata in the .opf file was not correctly patched, either on one or two occasions. 

\subsection{Computational Overhead}
A short overview of the computational overhead can be found in Table \ref{table:computational_overhead}. The execution times were found using a laptop with an AMD Ryzen 5 4600U processor while the system consumed on average 500MB of RAM. The table shows that an average EPUB takes under 3 seconds to be fully processed by AccessiPub.


\begin{table}[h!]
\small
\begin{center}
\begin{tabular}{ | c | c |} 
\hline
\textbf{Task} & \textbf{Execution time} \\
\hline
EPUB disassembly & 140 ms \\
\hline
Error finding and fixing & 100 ms \\
\hline
EPUB reassembly & 420 ms \\
\hline
EPUB patching & 2050 ms \\
\hline
\end{tabular}
\end{center}
\normalsize
\caption{Computational overhead in milliseconds per AccessiPub task for an average single EPUB}
\label{table:computational_overhead}
\end{table}
\section{Discussion}
\label{sec:discussion}
This section will expand on the results by comparing the findings to the existing research. Additionally, the limitations and constraints of the research are discussed, forming the foundation of potential areas for future work. 

\subsection{Comparison to existing research}
No other methods currently exist that deal with automating EPUB accessibility. Thus, the results are compared to research that uses HTML refactoring to improve accessibility on web pages, as this is the closest area of research. The target error reduction percentage is taken into account here, as the different research articles have their own methods of finding accessibility errors. Because of this, a comparison of the total error reduction percentage would be arbitrary.

Ferati and Sulejmani reached a target error reduction percentage of 48\%. Their methods were also built on HTML refactoring and focused on enriching links, images, and navigation. The accessibility errors were found using the aXe tool, a similar tool to the Ace checker from Daisy \cite{Ferati2016}.

Ikhsan and Candra were able to reduce the target errors in their research by between 53.6\% and 98.2\% depending on the website to which their tool was applied. This lead to an overall target error reduction of 75.4\%. They developed a proprietary HTML accessibility error finder that was able to find 1284 errors across five web pages, forming the data for the research \cite{Ikhsan2018}.

Zhang et al., aimed to add alt-text to SVG buttons that did not yet have alt-text. The study looked at 30 different public websites, and found 556 SVG buttons in need of an alt-text. In 511 of these cases, the methodology was able to correctly provide alt-text, giving a 91.9\% target error reduction. This alt-text was then added to the web page through HTML refactoring. \cite{Zhang2024}.

The methodology developed in this thesis was able to reduce the target errors by 97.8\%. Additionally, the reduction in errors was labeled to be statistically significant. In comparison with the other research articles occupying this field, AccessiPub fares exceptionally well. This demonstrates that this thesis neatly slots into the targeted research gap. It does this by successfully being the first venture into automatically refactoring EPUBs and enhancing their accessibility in a way that performs better than similar techniques on HTML files.


\subsection{Limitations and constraints}
This thesis also faced some notable limitations and constraints in the development of AccessiPub. These limitations were tied to the used libraries and tools, namely EbookLib and the Ace checker. In this subsection, these will be discussed, along with their impact on the results.

\subsubsection{EbookLib}
For EPUB disassembly and reassembly, the EbookLib library was chosen. Its powerful feature-set also allowed for page lists to be constructed for EPUBs that did not have one yet. However, it did bring some limitations with it. One major factor was that information in the \texttt{<head>} tag was lost, along with certain metadata, during EPUB reassembly. These limitations were circumvented as described in the \textit{EPUB patching} step of the Experimental Setup, but these workarounds proved to be time-intensive and did not cover all issues.

Additionally, the results were further impacted by EbookLib as 6 of the 9 unprocessed EPUBs were not processed because of EbookLib errors. Making use of a more simple library such as ZipFile can help with these specific issues, as it does not alter the source files. This option was not considered in this thesis due to time constraints, as the issues caused by EbookLib were not discovered until much later in the process. Porting the pagelist creation functionality from EbookLib to ZipFile would also require a lot of development time.

\subsubsection{Ace checker}
The Ace checker also causes some limitations for the results of this thesis. In some cases, the Ace checker fails to generate a valid report. 3 of the 9 unprocessed EPUBs remained unprocessed because the Ace checker could not handle the EPUB file that had images in the root folder. It is also important to mention that an Ace report without any reported errors does not necessarily mean that the EPUB is fully accessible. Not all accessibility tests are able to be automated. Despite this, the Ace checker remains state-of-the-art, because it is grounded in years of accessibility research and no comparison currently exists. Moreover, the Ace checker is still excellent at being able to make quantitative analyses possible for automating EPUB accessibility enhancement.
\section{Conclusion}
\label{sec:conclusion}
This thesis has made the first venture into the field of automating EPUB accessibility enhancement. We devised a bespoke methodology and named it AccessiPub. This methodology was composed of EPUB disassembly, error locating, HTML refactoring, EPUB reassembly, and EPUB patching. With the help of the Ace checker, the results were evaluated. 

This research answered its first three research questions which were methodological in nature. A strategy was developed for precisely locating accessibility errors within an EPUB. A new HTML refactoring technique was adapted from existing literature to fit EPUBs. Lastly, a method to reconstruct the modified content back into an EPUB was created. 

Then, the latter two research questions were answered through the experiments. The devised methodology reduced the total amount of accessibility errors in EPUBs by 83.6\% and reduced the amount of target errors by 97.8\%. Compared to similar research that focused on improving the accessibility of web pages, AccessiPub performs exceptionally well. These results were, to a small extent, limited by the limitations and constraints of EbookLib and the Ace checker, although their impact on the results was not hugely significant. Furthermore, the original integrity of the source EPUBs was largely upheld. Only 18 additional errors across 7 error types were introduced on top of the total of 128,194 errors after AccessiPub intervention. 

The results show the great potential of the possibilities of making EPUBs more accessible automatically. AccessiPub combines all the steps necessary for this process into a framework, which presents the main value of this research. This framework is able to be applied to many other EPUB accessibility problems, forging the path for future work. Only three errors were chosen for target errors, but now that a methodological framework is in place, other errors can be focused on in future work. As described in the Discussion, EbookLib formed the main limitation of this research. A promising path forward is for future research to explore a more simple and robust library for this purpose, such as ZipFile, as this will protect the original EPUB integrity.

\bibliographystyle{ACM-Reference-Format}
\bibliography{bibliographies/references}

\newpage
% You can choose whether you prefer a single or double column appendix.
% Whatever you choose, you will need to stick to it throughout the appendix.
% For double column style, comment the next line.
\onecolumn

\appendix
\begin{appendices}

\section{Appendix}
\label{sec:apx:first_appendix}

\begin{figure}[h]
\includegraphics[width=0.48\textwidth,keepaspectratio]{media/images/preaccessipub.png}
\caption{Distribution histogram of the amount of EPUBs per error count before AccessiPub intervention}
\centering
\label{figure:Appendix_pre}
\end{figure}

\begin{figure}[h]
\includegraphics[width=0.48\textwidth,keepaspectratio]{media/images/postaccessipub.png}
\caption{Distribution histogram of the amount of EPUBs per error count after AccessiPub intervention}
\centering
\label{figure:Appendix_post}
\end{figure}

\begin{figure}[h]
\includegraphics[width=0.72\textwidth,keepaspectratio]{media/images/ace-wrong.png}
\caption{Example 1a: Ace HTML report showing an unjustified link-in-text-block error}
\centering
\label{figure:Appendix_ace-wrong}
\end{figure}

\begin{figure}[h]
\includegraphics[width=0.72\textwidth,keepaspectratio]{media/images/ace-wrongv2.png}
\caption{Example 1b: Actual rendered HTML file showing an accessible link and showing that the Ace report is false}
\centering
\label{figure:Appendix_ace-wrong_code}
\end{figure}

\begin{figure}[h]
\includegraphics[width=0.72\textwidth,keepaspectratio]{media/images/again-ace-wrong.png}
\caption{Example 2a: Ace HTML report showing an unjustified link-in-text-block error}
\centering
\label{figure:Appendix_ace-wrong2}
\end{figure}

\begin{figure}[h]
\includegraphics[width=0.72\textwidth,keepaspectratio]{media/images/again-ace-wrongv2.png}
\caption{Example 2b: Actual rendered HTML file showing an accessible link and showing that the Ace report is false}
\centering
\label{figure:Appendix_ace-wrong2_code}
\end{figure}

\end{appendices}

\end{document}

%%%%%%%%%%%%%%%%%%%%%%%%%%%%%%%%%%%%%%%%%%%%%%%%%%%%%%%%%%%%%%%%%%%%%%%%%%%%%%%%
%%%%%%%%%%%%%%%%%%%%%%%%%%%%%%%%%%%%%%%%%%%%%%%%%%%%%%%%%%%%%%%%%%%%%%%%%%%%%%%%