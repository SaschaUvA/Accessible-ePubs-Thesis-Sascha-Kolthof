\section{Related Work}
\label{sec:related_work}
This thesis aims to address the existing research gap concerning the potential for automatically repairing EPUBs to be compliant with the WCAG. This section will explore the current research surrounding this topic by focusing on two topics. The first subsection is about the current guidelines posed by the most recent issue of the WCAG. The other subsection will focus on current methods for automatically increasing web content accessibility.

\subsection{Web Content Accessibility Guidelines}
The most recent version of the Web Content Accessibility Guidelines at the time of writing is 2.2. It is important to note that each version since 2.0 does not replace the old version, but rather adds additional guidelines on top of the former version. This means that, for instance 2.2 added nine new success criteria compared to 2.1 \cite{WCAGGuidelines}. The WCAG consists of four main principles: Perceivable, Operable, Understandable and Robust \cite{Caldwell2008}. Since new versions only apply incremental changes, the base structure and principles behind WCAG 2 have remained the same \cite{WCAGGuidelines}.

\textit{Perceivable} is the first principle listed by the WCAG and lays its focus on pursuing perceivable UI components and other web content. This includes guidelines, such as providing alternative text for images, making content distinguishable through color or contrast, and captions for time-based media \cite{Caldwell2008}.

Secondly, the WCAG state that navigating the web content and other UI components must be \textit{operable} to all. This principle is comprised of being able to control the web content by keyboard alone, providing enough time for users to read all text, and not having flashing elements that can cause a seizure \cite{Caldwell2008}.

The third principle of the WCAG is \textit{understandable}. What is meant by this, is that information on the page and UI components should be understandable for each user. This means, for example, that the web page should operate predictably to the user's input and that input fields are labeled when input is required \cite{Caldwell2008}.

Lastly, it is stated that content must be \textit{robust} to changes in order to be able to be offered to all users. This includes the usage of assistive technologies, that rely on the inner structure of the HTML files that make up the EPUB. This means that each element has opening and closing tags (e.g. <p> and </p>), and that all IDs are unique \cite{Caldwell2008}.

The WCAG also specifies conformance levels. These levels are A, AA and AAA, increasing in their level of accessibility demands. While level A is considered to be the minimum required level, it remains deficient in meeting the expectations of users with disabilities significantly. This is why level AA is generally used as the standard of accessibility \cite{KBDaisyWCAG}. The AAA conformance level does not fall within the scope of this thesis. This level includes success criteria such as prerecorded sign language, making this widely regarded to be superfluous in providing good accessibility, and only to be used supplementary \cite{WCAGGuidelines, KBDaisyWCAG}.

\subsection{Automatic methods for web accessibility}
This subsection will delve into the some of the existing literature on technologies for automatically improving accessibility on websites. Since EPUBs are in essence a collection of HTML files, this is the closest relevant and related research in this area.

Possible techniques for automatically improving accessibility on web pages is brought forward by Ferati \& Sulejmani \cite{Ferati2016}. Their research poses an automated tool that focuses on employing three techniques. Firstly, link enrichment will detect if a hyperlink is missing an appropriate label and add an \textit{aria-labelledby} attribute to the link in order to make it visible for screen readers \cite{Ferati2016}. Furthermore, the image enrichment will provide alt-text to images that do not have that already. On images that contained text, Optical Character Recognition (OCR) was applied to generate alt-text. In case OCR was not applicable to an image, the Google image API was used to fetch the description of similar images \cite{Ferati2016}. Lastly, Ferati \& Sulejmani applied navigation enrichment, which skips the navigation bar at the top of a web page. This way, screen readers will not continually repeat the navigation bar each time that the page is reloaded \cite{Ferati2016}. This combination of techniques was able to resolve over 50\% of all detected errors \cite{Ferati2016}.

Another approach is proposed by Ikhsan \& Candra, employing an automatic refactoring method \cite{Ikhsan2018}. Their approach consisted of a small number of steps, keeping the overall process simple. First, certain HTML tags were found using the HTML Code Sniffer. Then, this HTML tag would be checked for a certain condition. If this condition is met, the tag is compliant, otherwise, a pre-defined solution is applied to this tag. An example of such condition is, if a <head> tag is found without a title attribute, this attribute will be created \cite{Ikhsan2018}. This reduced the amount of errors by a percentage between 57\% and 96\% based on the website \cite{Ikhsan2018}. However, the research does not mention how, for instance, the content of the title attribute is created. It is thus possible that the overall accessibility of the web page did not improve significantly, in defiance of the significant reduction in errors.

Lastly, Zhang et al. makes use of the Beautiful Soup package in their approach. The objective of their research is to make SVG buttons on web pages more accessible by providing alt-text \cite{Zhang2024}. One of the challenges that Zhang et al. faced, was that the SVG buttons are hard to identify on a web page. This specific challenge was tackled using Beautiful Soup, employing its ability to find all potential SVG elements, and then sifting through these to find the legitimate SVG button. This strategy made it possible to provide alt-text to SVG buttons \cite{Zhang2024}.