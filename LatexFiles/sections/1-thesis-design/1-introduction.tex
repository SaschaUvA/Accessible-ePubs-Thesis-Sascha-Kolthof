\section{Introduction}
\label{sec:introduction}
% Mention scientific context/field, problem statement, research gap and candidate (sub) research question(s).
E-books are a great way of making regular books more portable and convenient to read. An EPUB (shorthand for \textit{electronic publication}) is an open-to-use file format for e-books, developed by the International Digital Publishing Forum (IDPF), that focuses on reflowable content. Reflowable content is the concept where alterations like increasing the font-size, make the page content flow dynamically to accommodate for this change. For people with special needs, this offers ways for books to be more accessible for them. For visually impaired people, EPUBs are also less cumbersome to read and transport than bulky braille books or books with large fonts. Additionally, there are certain accessibility tools that work best with EPUBs, making it the preferred option for people with disabilities \cite{Kasdorf2018}. This can include: text-to-speech devices (screen readers), math-specific screen readers, and braille displays \cite{Giusti2016, Kasdorf2018, Kim2019}. 

The COVID-19 pandemic has increased the amount of EPUB resources in libraries, simultaneously highlighting the issue with the current state of EPUB accessibility \cite{Chee2022}. Specifically, many EPUBs currently do not adhere to the latest accessibility standards posed by the Web Content Accessibility Guidelines (WCAG). While some research exists on what has to be done to improve the current accessibility landscape for EPUBs, it fails to go into how to make these improvements \cite{Chee2022}. Instead, articles propose a manual approach \cite{Campoverde2020}. More crucially for this thesis, current academia has not explored the potential role of automating the process of making EPUBs compliant on a large, cost-effective scale. 

This thesis aims to do a first venture into this specific area with the help of the Beautiful Soup 4 package in Python. The strategy, to be discussed in further detail later, will be to disassemble the EPUB files into its constituent parts (i.e. headers, photos, paragraphs), and reassemble these parts into a theoretically compliant framework that will be pre-defined using the most recent WCAG guidelines. This leads to the main research question in this proposed research: \\

\noindent\textit{To what extent can EPUBs automatically be repaired to comply to the WCAG 2.2 AA accessibility standards using Beautiful Soup 4 to reassemble the file's structure?} \\

\noindent This research question will be answered with the help of the following three sub questions:
\begin{itemize}
    \item What constitutes a WCAG 2.2 AA compliant EPUB?
    \item To what extent can an EPUB be disassembled into its constituent parts automatically?
    \item To what extent can a WCAG 2.2 AA compliant framework be created to serve as a mold for uncompliant EPUBs?
\end{itemize}