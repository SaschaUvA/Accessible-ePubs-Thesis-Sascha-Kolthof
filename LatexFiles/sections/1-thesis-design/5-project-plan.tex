\section{Project Plan}
\label{sec:project_plan}
% Describe a timeline via a Gantt chart or table with achievements per week.
As I will start working on the thesis in the week of March 11th, I will slightly deviate from the regular norm of 13 weeks, to 16 weeks. This extra time will be taken for more generous buffers to alleviate the thesis pressure as much as possible. One buffer will be placed after the tool development and edge cases focus, accommodating for the expected difficulties surrounding the edge cases. The second buffer is placed after the writing block to accommodate for any delays in writing. Below follows the Gantt Chart representing the schedule as it is now.

\newpage

% Gantt Chart
% For more complex Gantt charts see documentation here: 
% http://mirror.ox.ac.uk/sites/ctan.org/graphics/pgf/contrib/pgfgantt/pgfgantt.pdf
\begin{ganttchart}[
    expand chart=0.9\linewidth,
    vgrid,
    hgrid
    ]{0}{15}
        % Titles
        \gantttitle{Weeks}{16} \\
        \gantttitlelist{1,...,16}{1} \\

        % Group
        \ganttgroup{Data Aggregation}{0}{1} \\  % elem 0
        % Concrete tasks
        % More groups, further tasks ommitted
        \ganttgroup{Tool development}{2}{5} \\  % elem 4
        \ganttgroup{Edge cases}{6}{7} \\  % elem 5
        \ganttgroup{Buffer 1}{8}{9} \\ % elem6
        \ganttgroup{Evaluation}{10}{10} \\  % elem 7
        \ganttmilestone{Finish Experiments}{10} \ganttnewline % elem 8
        \ganttgroup{Writing}{11}{14} \\  % elem 9
        \ganttgroup{Buffer 2}{15}{15} % elem 10
        
        % Connectors
        \ganttlink{elem0}{elem1}
        \ganttlink{elem1}{elem2}
        \ganttlink{elem2}{elem3}
        \ganttlink{elem3}{elem4}
        \ganttlink{elem4}{elem5}
        \ganttlink{elem5}{elem6}
        \ganttlink{elem6}{elem7}
  \label{ganttchart}
\end{ganttchart}
