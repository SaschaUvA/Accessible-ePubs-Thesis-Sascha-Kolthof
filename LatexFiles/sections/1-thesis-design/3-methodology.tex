\section{Methodology}
\label{sec:methodology}
The methodology section of this thesis design is split into two subsections. The first subsection will delve into the current available data, ongoing efforts to gather more data, and a description of the EPUB file type. The second subsection will focus on the current strategy for repairing EPUBs and the accessibility assessment tool that will assume a central role in this thesis.

\subsection{Data}
The data for this thesis project currently consists of 944 EPUBs, sourced by the \textit{OAPEN} foundation (Open Access Publishing in European Networks). The average number of pages in this collection is 270 when rounded to a whole number, and has a standard deviation of 82 pages.

Since this project is performed with multiple people, we are still finding more ways to accumulate more data for. One example of this endeavour, is finding the \textit{Standard Ebooks L3C} collection. This is a crowd-funded project that focuses on making literature available that has already passed their copyright status (e.g. Pride and Prejudice). This dataset consists of 953 e-books of varying size. While the bulk-download of this dataset is behind a paywall, individual EPUB downloads are available to any user of the website. A script that can scrape these 953 e-books using the individual method is under development.

Additionally, conversations are ongoing with the Dutch Royal Library to acquire more data. It is paramount for this project to acquire as much data as possible, as this will highlight as many edge cases as possible. Solving these edge cases will provide a more robust EPUB repair tool in the end. 

These edge cases are likely to occur in the structure of the EPUB. In essence an EPUB is a ZIP folder with a .epub file extension that holds a folder with metadata and a folder with page contents. In the page contents, the images, text and CSS stylesheet are found. The text is divided into chapters comprised of .xhtml files. The structure and layout of the content folders, and more importantly of the HTML files can vary greatly between EPUBs. This variety is assumed to become the main source of edge cases, creating the main challenge of this project.

\subsection{Workflow \& evaluation}
Before the EPUBs can be repaired, they first have to be checked to see what is inaccessible about them. This will be done using the \textit{Ace} checker. This open-source free tool was developed by \textit{The Daisy Consortium}. It checks an EPUB for breaches of built-in rulesets. These rulesets and their specifics are documented in its entirety on the project's Github page, but are for the most part built on the WCAG \cite{AceRules}. The version of Ace that will be used in this project, is version 1.3.2, which is the latest version at the time of writing.

Since the bulk of an EPUB is comprised of HTML files, the text data will be repaired using the Beautiful Soup 4 Python package. This package is especially designed for extracting data from an HTML file, and is equipped to handle different HTML structures. It is thus an exceptional fit for dealing with the varying structure and text contents of the EPUB. The current strategy of applying this package is to first extract all relevant content from the text and stylesheet. This content will then be fitted into a separate pre-prepared structure and stylesheet that already complies up to the WCAG 2.2 AA guidelines. Once the content is fitted into this structure successfully, the EPUB will be compliant to the set guidelines. 

The goal of this project is that all success criteria up until the WCAG 2.2 AA guidelines are met after an EPUB has been processed. The EPUBs will be checked with the Ace checker after being repaired to evaluate the performance of the tool. The checker catalogues any breach of the ruleset as Critical, Serious, Moderate or Minor. If no such errors remain, the EPUB will be fully WCAG 2.2 AA compliant. If errors do still remain, the amount of errors will be compared to the EPUB before the intervention to evaluate the performance of the tool. This secondary goal is to greatly reduce the Critical, Serious, Moderate and Minor rule breaches of the EPUB automatically.