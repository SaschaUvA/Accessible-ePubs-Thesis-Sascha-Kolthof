\section{Risk Assessment}
\label{sec:risk_assessment}
% Describe the risks which you could run into and how you will mitigate them.
The biggest risk that this thesis project will face is that the developed tool at the end does not work on all EPUBs. This is a realistic scenario, because the extent of what all edge cases look like is not easily judged. This will be influenced by the size and variety of the dataset not being high enough to find all possible edge cases. Finding as many edge cases as possible might cost more time than initially thought. Although at some point, it must be accepted that not all EPUBs will be fully repairable, as solving all possible edge cases is not a feasible goal.

Since this thesis project has multiple people assigned to the same topic, the loss of a group member is a realistic scenario. The project is not dependant on all people finishing, and a loss of one member is not expected to cause any significant time loss. However, this might have an impact on the amount and variety of data gathered, impeding some of the ambitious goals of this project that rely on this shared pool of resources and knowledge.

Modifying EPUBs that are protected under copyright laws might not be possible without authorization. This means that all additional data be free of copyright. This will slow down additional data gathering endeavours, impacting the size and variety of the dataset.