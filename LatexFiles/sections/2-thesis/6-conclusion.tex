\section{Conclusion}
\label{sec:conclusion}
This thesis has made the first venture into the field of automating EPUB accessibility enhancement. We devised a bespoke methodology and named it AccessiPub. This methodology was composed of EPUB disassembly, error locating, HTML refactoring, EPUB reassembly, and EPUB patching. With the help of the Ace checker, the results were evaluated. 

This research answered its first three research questions which were methodological in nature. A strategy was developed for precisely locating accessibility errors within an EPUB. A new HTML refactoring technique was adapted from existing literature to fit EPUBs. Lastly, a method to reconstruct the modified content back into an EPUB was created. 

Then, the latter two research questions were answered through the experiments. The devised methodology reduced the total amount of accessibility errors in EPUBs by 83.6\% and reduced the amount of target errors by 97.8\%. Compared to similar research that focused on improving the accessibility of web pages, AccessiPub performs exceptionally well. These results were, to a small extent, limited by the limitations and constraints of EbookLib and the Ace checker, although their impact on the results was not hugely significant. Furthermore, the original integrity of the source EPUBs was largely upheld. Only 18 additional errors across 7 error types were introduced on top of the total of 128,194 errors after AccessiPub intervention. 

The results show the great potential of the possibilities of making EPUBs more accessible automatically. AccessiPub combines all the steps necessary for this process into a framework, which presents the main value of this research. This framework is able to be applied to many other EPUB accessibility problems, forging the path for future work. Only three errors were chosen for target errors, but now that a methodological framework is in place, other errors can be focused on in future work. As described in the Discussion, EbookLib formed the main limitation of this research. A promising path forward is for future research to explore a more simple and robust library for this purpose, such as ZipFile, as this will protect the original EPUB integrity.